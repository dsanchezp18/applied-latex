\documentclass[a4paper,11pt]{article}

\usepackage[utf8]{inputenc}
\usepackage[american]{babel}
\usepackage{csquotes}
\usepackage{hyperref}
\usepackage[margin=1in]{geometry}
\usepackage[style = windycity, backend=biber]{biblatex}
    \addbibresource{referencias_temario.bib}
    \nocite{*}

\title{Applied \LaTeX \ and Markdown for Researchers}
\author{Laboratorio de Investigación para el Desarrollo del Ecuador}
\date{Syllabus}

\begin{document}

\maketitle

\noindent \textbf{Instructor}: Daniel Sánchez, M.A. \\
\textbf{Module length}: 4 hours \\
\textbf{Level}: Introductory \\
\textbf{GitHub repository}: \url{https://github.com/laboratoriolide/applied-latex}

\section{Course Description}

This short module will introduce the use of the typographic system \LaTeX, focusing on its applied use for research. Further, the short module introduces Markdown and its use in conjunction with data analysis software. 

\section{Contents}

The following is a planned outline of the course. This may change depending on the pace of the
class. Lecture materials will be uploaded to the module's \href{https://github.com/laboratoriolide/applied-latex}{GitHub repository}.

\subsection{Lecture 1: Introduction to \LaTeX \ and document editing}

\begin{itemize}
    \item Introduction to \LaTeX, what is it and what is it for?
    \item Brief history of \LaTeX
    \item Preliminary issues
    \begin{itemize}
        \item Hardware requirements
        \item Installation of \TeX \ distributions
        \item Development environments (IDEs: VS Code, \TeX Maker, etc.)
        \item Overleaf: using \LaTeX \ online
        \item Identification of keyboard shortcuts and important keycaps
    \end{itemize}
\end{itemize}

\begin{itemize}
    \item \LaTeX \ file structure
    \item Simple commands
    \item Packages
    \item Document classes
    \item Basic document formatting
    \item Text handling
    \begin{itemize}
        \item Alignment
        \item Lists
        \item Titles, covers and abstracts
        \item Indexes
        \item Headers and footnotes 
    \end{itemize}
\end{itemize}

\subsection{Lecture 2: Math mode, tables and figures}

\begin{itemize}
    \item Introduction to math mode
    \item Basic symbols and greek letters
    \item Equations
    \item Matrices
    \item Basic tables
    \item Automated table-making: Excel2LaTeX / Overleaf addins
    \item Including figures and subfigures
\end{itemize}

\subsection{Lecture 3: Bibliography management with \LaTeX, complex documents}

\begin{itemize}
    \item Bib\TeX \ and Bib\LaTeX
    \item Zotero integration
    \item Brief review of Mendeley, Citavi and other integrations
    \item Citation and bibliography formatting, styling
    \item Multi-file projects
    \item Cross-referencing
    \item Tips for error debugging
    \item Time-permitting: basic plotting with \textsf{tikz} and \textsf{pgfplots}
\end{itemize}

\subsection{Lecture 4: Integration with statistical packages}

\begin{itemize}
    \item RMarkdown/Quarto
    \begin{itemize}
        \item Basic Markdown syntax
        \item R code chunks
        \item Output formats
        \item Use of \LaTeX
    \end{itemize}
    \item Presenting data analysis results with R
    \begin{itemize}
        \item \textit{stargazer}
        \item \textit{kableExtra}
        \item \textit{modelsummary}
        \item \textit{gt} and \textit{flextable}
    \end{itemize}
    \item Stata
    \begin{itemize}
        \item \textit{estout}
        \item \textit{outreg2}
    \end{itemize}
\end{itemize}

\subsection{Advanced topics (if time allows)}

\begin{itemize}
    \item Presentations with beamer and Quarto
    \item Advanced document formatting with classes
    \item Using \LaTeX \ from Word
    \item \textit{knitr/sweave}
    \item Codecogs
    \item Working with Python/Jupyter
\end{itemize}

\section{Evaluation}

Please consult the program's regulation manual for short module evaluation criteria. All communication will be done through the program's Slack channel. 

\section{Materials}

I will provide all necessary materials for the course in the GitHub repository. Typically, this will include lecture slides, code snippets, and other resources. Consult the Reference Material section for additional resources. 

\section{Software}

We will mostly rely on Overleaf. However, you must install a \TeX \ distribution on your computer for local demonstrations. I recommend using \href{https://www.tug.org/texlive/}{\TeX Live} for Windows, Linux and macOS. An alternative is \href{https://miktex.org/}{MiK\TeX} for Windows.

For editing, an integrated development environment (IDE) is recommended. \TeX \ Maker is a good option for Windows, macOS and Linux.

For Markdown, we will use VS Code and RStudio, both of which are free. You can install the \textit{Markdown All in One} extension for VS Code. Follow the instructions for your *Intro to R* module to install R and RStudio. For cloning the repository, you will need Git, which you can install with your instructions from the *Git \& GitHub* module.

\section{Keyboard layout}

We will routinely need to type symbols like and others. Make sure you are comfortable with your keyboard layout and that you can type these symbols easily. This may seem trivial, but it is important for the course, as we absolutely cannot afford to lose time finding symbols on the keyboard. You may need to change your keyboard layout to the correct language so that the computer follows the physical layout of your keyboard. For Windows users, this is easily done by pressing \texttt{Win + Space} and selecting the correct layout (see \href{https://support.microsoft.com/en-us/windows/change-your-keyboard-layout-245c49b8-f856-7fd7-2cf5-41e54c66f5b3}{here})\footnote{Mac and Linux Users, sorry, you're on your own here.}.

\section{Communication}

All communication will be done through the program's Slack channel. I do not monitor email regularly, so please use Slack for any questions or concerns.

\renewcommand\bibname{Reference Material}
\printbibliography

\end{document}
