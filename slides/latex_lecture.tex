\documentclass[t,12pt,xcolor=dvipsnames]{beamer}
\usepackage{setspace}
\usepackage[canadian]{babel}
\setstretch{1.5}
\usepackage[utf8]{inputenc}
\usepackage{nicefrac}
\usepackage{biblatex}
\usepackage{amsmath}
\usepackage{csquotes}
\usetheme{Berlin}
\useoutertheme{infolines}
\useinnertheme{rounded}
\definecolor{darkergreen}{rgb}{0.12, 0.3, 0.17}
\definecolor{lapislazuli}{rgb}{0.15, 0.38, 0.61}
% Define a beige color used for some highlights
\definecolor{beige}{rgb}{0.96, 0.96, 0.86}
\definecolor{black}{rgb}{0,0,0}
\definecolor{burlywood}{rgb}{0.87, 0.72, 0.53}
\setbeamercolor{background canvas}{bg=white}

\usecolortheme[named=black]{structure}
%\definecolor{cambridgeblue}{rgb}{0.64, 0.76, 0.68}
%\definecolor{lavenderblue}{rgb}{0.8, 0.8, 1.0}

\usepackage{amssymb,amsmath,tabu}
\usepackage{graphicx}
\usepackage{booktabs}
\usepackage{multirow}

\title[Applied \LaTeX]{Applied \LaTeX \ for Researchers}
\author[DS]{Daniel Sánchez, M.A.}
\institute{Workshops for Ukraine}
\date{March 2025}

\begin{document}
\begin{frame}
    \titlepage
\end{frame}

\begin{frame}{What is \LaTeX \ ? Why even care?}
    \begin{itemize}
    \item A typesetting system, widely used in academia.
    \item Allows for additional control over the structure and layout of documents other software does not easily provide.
    \item Free, open-source, and cross-platform.
    \item \textit{What you see is what you mean} (WYSIWYM) vs. \textit{What you see is what you get} (WYSIWYG). % Separate content from presentation. 
    % word 
    \item Allows for the creation and automation of complex, structured and consistent documents.
    \end{itemize}
\end{frame}

\begin{frame}{Outline}
    \tableofcontents
\end{frame}

\section{Getting started}

%\begin{frame}{Brief History}
    %\begin{itemize}
        %\item Created by Leslie Lamport in 1983 while working at Stanford Research Institute.
        %\item Based on Donald Knuth's \TeX \ typesetting system (1978).
        %\item \LaTeX \ is a set of macros for \TeX.
        %\item Current version is \LaTeXe, released in 1994, replacing \LaTeX 2.09.
    %\end{itemize}
%\end{frame}

\begin{frame}{Important keys}
    \begin{itemize}
        \item \textbf{Command keys:} \textbackslash
        \item \textbf{Curly braces:} \{ \}
        \item \textbf{Square brackets:} [ ]
        \item \textbf{Percent sign:} \% (comments)
        \item \textbf{Dollar sign:} \$ (math mode)
        \item \textbf{Underscore:} \_
        \item \textbf{Circumflex:} \^{}
        \item \textbf{Tilde:} \~{}
        \item \textbf{Backslash:} \textbackslash
\end{itemize}
\end{frame}

\begin{frame}{Using \LaTeX \ online}
    \begin{itemize}
        \item Overleaf is a popular online \LaTeX \ editor.
        \item Share projects with collaborators.
        \item Real-time collaboration.
        \item Access to a wide range of templates.
        \item Free and paid versions.
        \item IMO great for starters and probably the best option for collaborative work and best-looking UI.
    \end{itemize}
\end{frame}

\begin{frame}{Using \LaTeX \ locally}
    \begin{itemize}
        \item You will need a \TeX \ distribution.
        \item For Windows, Mik\TeX \ is a popular choice, or \TeX \ Live.
        \item A \LaTeX \ editor will also be needed.
        \begin{itemize}
            \item \TeX Works
            \item \TeX Maker
            \item \TeX Studio
            \item Sublime Text
            \item VS Code
        \end{itemize}
    \end{itemize}
\end{frame}

\section{Beginning a document}

\begin{frame}{Basic structure}
    \begin{itemize}
        \item A \LaTeX \ document is divided into two main parts: the preamble and the body.
        \item The preamble contains document-wide settings and commands.
        \item The body contains the content of the document (text, figures, tables, etc.).
        \item The document is enclosed in the \texttt{document} environment.
    \end{itemize}
\end{frame}

\begin{frame}{Document preamble}
    \begin{itemize}
        \item The preamble is the first part of the document, containing configuration settings for the complete document.
        \item Technically, it is a \TeX \ environment (more on that later)
        \item We can set the document class, font size, margins, packages, etc.
        \item The preamble is enclosed between \texttt{documentclass} and \texttt{begin\{document\}} commands.
    \end{itemize}
\end{frame}

\begin{frame}{Writing commands/code}
    \begin{itemize}
        \item Commands start with a backslash (\textbackslash).
        \item Commands can have arguments enclosed in curly braces (\{ \}).
        \item Some commands have optional arguments enclosed in square brackets ([ ]).
        \item Comments are preceded by a percent sign (\%).
        \item Commands are case-sensitive.
    \end{itemize}
\end{frame}

\begin{frame}{Declaring the document class}
    \begin{itemize}
        \item The document class defines the overall layout of the document.
        \item The most common document classes are \texttt{article}, \texttt{report}, \texttt{book}, and \texttt{beamer}.
        \item Declared with the \texttt{documentclass} command.
        \item We will typically work with the \texttt{article} class.
    \texttt{\textbackslash documentclass\{article\}}
    \end{itemize}
\end{frame}

\begin{frame}{Some basic options for the document class}
    \begin{itemize}
        \item \texttt{10pt, 11pt, 12pt}: Sets the font size. Default is 10pt.
        \item \texttt{a4paper, letterpaper}: Sets the paper size.
        \item Other options are available depending on the document class.
        \item Packages can be used to extend the functionality of the document class.
        \item Use comments to keep track of the options used!
    \end{itemize}
\end{frame}
\begin{frame}{The document environment}
    \begin{itemize}
        \item The \texttt{document} environment is where the content of the document is placed.
        \item It is enclosed within the \texttt{begin\{document\}} and \texttt{end\{document\}} commands.
        \item  All \LaTeX \ environments require a \texttt{begin} and \texttt{end} command.
        \item Nothing should be written after the \texttt{end\{document\}} command.
        \item Before the \texttt{begin\{document\}} command, we have the preamble.
    \end{itemize}
\end{frame}

\begin{frame}{Writing some text}
    \begin{itemize}
        \item Text is written directly in the document environment.
        \item \LaTeX \ ignores multiple spaces. 
        \item Use the \texttt{\textbackslash\textbackslash} command to start a new line.
        \item Use the \texttt{\textbackslash par} command to start a new paragraph.
        \item Use the \texttt{\textbackslash newline} command to start a new line.
    \end{itemize}
\end{frame}

\begin{frame}{Basic text management}
    \begin{itemize}
        \item To italicize text, use the \texttt{\textbackslash textit\{\}} command.
        \item To bold text, use the \texttt{\textbackslash textbf\{\}} command.
        \item To underline text, use the \texttt{\textbackslash underline\{\}} command.
        \item To change the font size, use the \texttt{\textbackslash tiny}, \texttt{\textbackslash small}, \texttt{\textbackslash large}, etc. commands.
    \end{itemize}
\end{frame}

\begin{frame}{Special characters}
    \begin{itemize}
    \item Some characters have special meanings in \LaTeX, hence, they need to be escaped to be printed in the document.
    \item The following characters are reserved: \texttt{\# \$ \% \^{} \& \_ \{ \} \~{}}.
    \item To print these characters, use the \texttt{\textbackslash\#}, \texttt{\textbackslash\$}, \texttt{\textbackslash\%}, etc. commands.
    \item The \texttt{\textbackslash} itself is printed with \texttt{\textbackslash textbackslash}.
    \end{itemize}
\end{frame}

\begin{frame}{Structure}
    \begin{itemize}
        \item \LaTeX \ provides commands to structure the document with sections, subsections, and subsubsections.
        \item It is generally a good idea to use these commands to organize the content of the document.
        \item \LaTeX-produced PDFs generally bookmark the sections, making navigation easier (if the viewer supports it and a package is used, more on that later).
    \end{itemize}
\end{frame}
\begin{frame}{Structure commands}
    \begin{itemize}
        \item \texttt{\textbackslash section\{Section title\}}
        \item \texttt{\textbackslash subsection\{Subsection title\}}
        \item \texttt{\textbackslash subsubsection\{Subsubsection title\}}
        \item \texttt{\textbackslash paragraph\{Paragraph title\}}
        \item \texttt{\textbackslash subparagraph\{Subparagraph title\}}
    \end{itemize}
\end{frame}

\begin{frame}{Titles}
    \begin{itemize}
        \item In the preamble of the document, one can define \enquote{document metadata} such as the title, author, and date.
        \item This information can be printed in the document using the \texttt{\textbackslash maketitle} command.
        \item Use the \texttt{\textbackslash title\{\}}, \texttt{\textbackslash author\{\}}, and \texttt{\textbackslash date\{\}} commands to define the metadata.
        \item The title commands are largely determined by the document class, but can be customized with packages
    \end{itemize}
\end{frame}
\begin{frame}{Packages}
    \begin{itemize}
        \item Packages are used to extend the functionality of the document class.
        \item They can be loaded in the preamble with the \texttt{\textbackslash usepackage\{\}} command.
        \item Some packages are included by default in the document class.
        \item Some common packages are \texttt{graphicx}, \texttt{amsmath}, \texttt{hyperref}, \texttt{babel}, \texttt{inputenc}, \texttt{fontenc}, \texttt{geometry}, \texttt{fancyhdr}, among others.
    \end{itemize}
\end{frame}

\begin{frame}{Common packages}
    \begin{itemize}
        \item The \texttt{geometry} package can be used to set the margins of the document.
        \item The \texttt{setspace} package can be used to set the line spacing (single, 1.5, double).
        \item The \texttt{lipsum} package can be used to generate dummy text. 
    \end{itemize}
\end{frame}

\begin{frame}{Lists}
    \begin{itemize}
        \item Two main lists are commonly used in \LaTeX: \texttt{itemize} and \texttt{enumerate}.
        \item The \texttt{itemize} environment is used for unordered lists.
        \item The \texttt{enumerate} environment is used for ordered lists.
        \item Items are declared with the \texttt{\textbackslash item} command.
        \item Nested lists are possible, and the \texttt{description} environment can be used for descriptions for all items.
    \end{itemize}
\end{frame}

\section{Math mode}

\begin{frame}{Why math mode?}
    \begin{itemize}
        \item \LaTeX's math mode is where it truly shines.
        \item This syntax has become a standard for typesetting math, even beyond \LaTeX.
        \item While Microsoft Word's equation editor has come a long way, when things get complex, \LaTeX \ is the way to go
        \begin{itemize}
            \item Aligning equations
            \item Repetitive notation
            \item Lemmas, theorems, proofs
            \item Complex symbols
        \end{itemize}
    \end{itemize}
\end{frame}

\begin{frame}{Enter Math Mode}
    \begin{itemize}
        \item Math mode is entered with the \texttt{\$} symbol.
        \item Inline math mode is entered with a single \texttt{\$} symbol.
        \item Display math mode is entered with double \texttt{\$} symbols. This will center the equation and do an automatic line break.
    \end{itemize}
            $$ \sum_{i=1}^{n} i = \frac{n(n+1)}{2} $$
\end{frame}

\begin{frame}{Some simple algebra}
    \begin{itemize}
        \item Math mode changes the font of the text to the \enquote{math font}.
        \item To create exponents, use the \texttt{\^} symbol (\enquote{caret} or \enquote{hat}).
        \item To create subscripts, use the \texttt{\_} symbol (\enquote{underscore}).
        \item To create fractions, use the \texttt{\textbackslash frac\{\}\{\}} command.
        \item To create square roots, use the \texttt{\textbackslash sqrt\{\}} command
    \end{itemize}
    $$ x_1^2 + y_2^2 = z_1^2 $$
\end{frame}
\begin{frame}{Greek letters}
    \begin{itemize}
        \item Greek letters can be written in math mode using their name preceded by a backslash.
        \item You will need the \texttt{amsmath} package for some of these (the package includes the \texttt{amssymb} package).
        \item They are case sensitive. To write the uppercase version, capitalize the first letter of the command. For example, \texttt{\textbackslash delta} for $\delta$ and \texttt{\textbackslash Delta} for $\Delta$.
        \item Some Greek letters have variants, such as \texttt{\textbackslash varphi} and \texttt{\textbackslash phi}.
    \end{itemize}
\end{frame}

\begin{frame}{Summations and integrals}
    \begin{itemize}
        \item To write a summation, use the \texttt{\textbackslash sum} command.
        \item To write an integral, use the \texttt{\textbackslash int} command.
        \item Since these have limits, use the \texttt{\_} and \texttt{\^} symbols to denote the lower and upper limits. They are like subscripts and superscripts!
    \end{itemize}
    $$ \int x^2 dx = \frac{x^3}{3} + C $$
\end{frame}

\begin{frame}{Using text within math mode}
    \begin{itemize}
        \item If you try to write text within math mode, it will be printed in the math font.
        \begin{itemize}
            \item For example, $ \text{where} $ will be printed as $ where $.
            \item Spacing and formatting will be weird
        \end{itemize}
        \item To write text within math mode, use the \texttt{\textbackslash text\{\}} command.
        \item This will change the font back to the regular text font.
        \item This is useful for writing text within equations, such as units.
    \end{itemize}
    $$ \ln(\text{income}) = \beta_0 + \beta_1 \text{education} + \text{other stuff} +  \epsilon $$
\end{frame}

\begin{frame}{Math environments}
    \begin{itemize}
        \item Many math environments are available in \LaTeX, which can be used to align math expressions. 
        \item The \texttt{align} environment is one of the most useful.
        \item Additionally, the \texttt{equation} environment can be used to number equations.
    \end{itemize}
\end{frame}

\begin{frame}{Alignment of equations}
\begin{itemize}
    \item The \texttt{align} environment is used to align equations.
    \item The \texttt{\&} symbol is used to specify the alignment point
    \item A double backslash (\texttt{\textbackslash\textbackslash}) is used to start a new line.
    \item The \texttt{align*} environment can be used to suppress equation numbering.
    \item Double \& symbols can be used to align multiple points. 
    \item The \texttt{quad} command can be used to add space between equations.
\end{itemize}
\end{frame}

\begin{frame}{Creating tables}
    \begin{itemize}
        \item Tables are created using the \texttt{tabular} environment.
        \item Columns are defined using the \texttt{l}, \texttt{c}, and \texttt{r} specifiers for left, center, and right alignment, respectively.
        \item Columns are separated by the \texttt{\&} symbol.
        \item Rows are separated by the double backslash (\texttt{\textbackslash\textbackslash}) command.
        \item Horizontal lines can be added using the \texttt{\textbackslash hline} command.
    \end{itemize}
\end{frame}

\begin{frame}{Example of a simple table}
    \centering
    \begin{tabular}{|c|c|c|}
        \hline
        Column 1 & Column 2 & Column 3 \\
        \hline
        Data 1 & Data 2 & Data 3 \\
        Data 4 & Data 5 & Data 6 \\
        Data 7 & Data 8 & Data 9 \\
        \hline
    \end{tabular}
\end{frame}

\begin{frame}{Advanced tables}
    \begin{itemize}
        \item The \texttt{booktabs} package provides additional commands for creating professional-looking tables.
        \item The \texttt{\textbackslash toprule}, \texttt{\textbackslash midrule}, and \texttt{\textbackslash bottomrule} commands can be used to add horizontal lines.
        \item The \texttt{multirow} package allows for cells to span multiple rows.
        \item The \texttt{tabu} package provides additional functionality for creating complex tables.
    \end{itemize}
\end{frame}

\begin{frame}{Example of an advanced table}
    \centering
    \begin{tabular}{|c|c|c|}
        \toprule
        Column 1 & Column 2 & Column 3 \\
        \midrule
        \multirow{2}{*}{Data 1} & Data 2 & Data 3 \\
        & Data 5 & Data 6 \\
        Data 7 & \multicolumn{2}{c|}{Data 8 and 9} \\
        \bottomrule
    \end{tabular}
\end{frame}

\begin{frame}{Tools for creating tables}
    \begin{itemize}
        \item Manually creating the \texttt{tabular} environment can be tedious.
        \item Check out online tools such as the Overleaf tables editor or \href{tablesgenerator.com}{tablesgenerator.com}.
        \item The Excel add-in, Excel2LaTeX, can be used to convert Excel tables to \LaTeX, \href{https://github.com/ivankokan/Excel2LaTeX/releases/tag/v3.5.0}{download here}.
    \end{itemize}
\end{frame}

\begin{frame}{Figures}
    \begin{itemize}
        \item Figures can be included in a \LaTeX \ document using the \texttt{figure} environment.
        \item The \texttt{graphicx} package is used to include images.
        \item The \texttt{\textbackslash includegraphics} command is used to include images.
        \item The \texttt{caption} command is used to add a caption to the figure.
        \item The \texttt{label} command is used to add a label to the figure for cross-referencing.
        \item Tips: take care of file paths!
    \end{itemize}
\end{frame}

%\section{Citations and References}

%\section{Exporting to \LaTeX from other software}

\end{document}
