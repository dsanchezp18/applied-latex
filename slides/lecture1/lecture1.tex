\documentclass[t,12pt,xcolor=dvipsnames]{beamer}
\usepackage{setspace}
\usepackage[canadian]{babel}
\setstretch{1.5}
\usepackage[utf8]{inputenc}
\usepackage{nicefrac}
\usepackage{biblatex}
\usetheme{Berlin}
\useoutertheme{infolines}
\useinnertheme{rounded}
\definecolor{darkergreen}{rgb}{0.12, 0.3, 0.17}
\definecolor{lapislazuli}{rgb}{0.15, 0.38, 0.61}
\definecolor{veige}{rgb}{0.12,,215}
\definecolor{black}{rgb}{0,0,0}
\definecolor{burlywood}{rgb}{0.87, 0.72, 0.53}
\setbeamercolor{background canvas}{bg=white}


\usecolortheme[named=black]{structure}
%\definecolor{cambridgeblue}{rgb}{0.64, 0.76, 0.68}
%\definecolor{lavenderblue}{rgb}{0.8, 0.8, 1.0}

\usepackage{amssymb,amsmath,tabu}
\usepackage{graphicx}
\usepackage{booktabs}
\usepackage{multirow}

\title[Applied \LaTeX]{Applied \LaTeX \ for Researchers}
\subtitle{Lecture 1: Introduction and Basics}
\author{Daniel Sánchez, M.A.}
\institute[LIDE]{Laboratorio de Investigación para el Desarrollo del Ecuador}
\date{Spring 2025}

\begin{document}
\begin{frame}
    \titlepage
\end{frame}



\begin{frame}{What is \LaTeX \ ? Why even care?}
    \begin{itemize}
    \item A typesetting system, widely used in academia.
    \item Allows for additional control over the structure and layout of documents other software does not easily provide.
    \item Free, open-source, and cross-platform.
    \item \textit{What you see is what you mean} (WYSIWYM) vs. \textit{What you see is what you get} (WYSIWYG). % Separate content from presentation.
    \item Allows for the creation and automation of complex, structured and consistent documents.
    \end{itemize}
\end{frame}

\begin{frame}{Outline}
    \tableofcontents
\end{frame}

\section{Getting started}

\begin{frame}{Brief History}
    \begin{itemize}
        \item Created by Leslie Lamport in 1983 while working at Stanford Research Institute.
        \item Based on Donald Knuth's \TeX \ typesetting system (1978).
        \item \LaTeX \ is a set of macros for \TeX.
        \item Current version is \LaTeXe, released in 1994, replacing \LaTeX 2.09.
    \end{itemize}
\end{frame}

\begin{frame}{Important keys}
    \begin{itemize}
        \item \textbf{Command keys:} \textbackslash
        \item \textbf{Curly braces:} \{ \}
        \item \textbf{Square brackets:} [ ]
        \item \textbf{Percent sign:} \% (comments)
        \item \textbf{Dollar sign:} \$ (math mode)
        \item \textbf{Underscore:} \_
        \item \textbf{Circumflex:} \^{}
        \item \textbf{Tilde:} \~{}
        \item \textbf{Backslash:} \textbackslash
\end{itemize}
\end{frame}

\begin{frame}{Using \LaTeX locally}
    \begin{itemize}
        \item You will need a \TeX \ distribution.
        \item For Windows, Mik\TeX \ is a popular choice, or \TeX \ Live.
        \item A \LaTeX editor will also be needed.
        \begin{itemize}
            \item \TeX Works
            \item \TeX Maker
            \item \TeX Studio
            \item Sublime Text
            \item VS Code
        \end{itemize}
    \end{itemize}
\end{frame}

\begin{frame}{Using \LaTeX online}
    \begin{itemize}
        \item Overleaf is a popular online \LaTeX \ editor.
        \item Share projects with collaborators.
        \item Real-time collaboration.
        \item Access to a wide range of templates.
        \item Free and paid versions.
        \item IMO great for starters and probably the best option for collaborative work and best-looking UI.
    \end{itemize}
\end{frame}

\section{Beginning a document}

\begin{frame}{Basic structure}
    \begin{itemize}
        \item A \LaTeX \ document is divided into two main parts: the preamble and the body.
        \item The preamble contains document-wide settings and commands.
        \item The body contains the content of the document (text, figures, tables, etc.).
        \item The document is enclosed in the \texttt{document} environment.
    \end{itemize}
\end{frame}

\begin{frame}{Document preamble}
    \begin{itemize}
        \item The preamble is the first part of the document, containing configuration settings for the complete document.
        \item Technically, it is a \TeX environment (more on that later)
        \item We can set the document class, font size, margins, packages, etc.
        \item The preamble is enclosed between \texttt{documentclass} and \texttt{begin\{document\}} commands.
    \end{itemize}
\end{frame}

\begin{frame}{Writing commands/code}
    \begin{itemize}
        \item Commands start with a backslash (\textbackslash).
        \item Commands can have arguments enclosed in curly braces (\{ \}).
        \item Some commands have optional arguments enclosed in square brackets ([ ]).
        \item Comments are preceded by a percent sign (\%).
        \item Commands are case-sensitive.
    \end{itemize}
\end{frame}

\begin{frame}{Declaring the document class}
    \begin{itemize}
        \item The document class defines the overall layout of the document.
        \item The most common document classes are \texttt{article}, \texttt{report}, \texttt{book}, and \texttt{beamer}.
        \item Declared with the \texttt{documentclass} command.
        \item We will typically work with the \texttt{article} class.
    \texttt{\textbackslash documentclass[12pt]\{article\}}
    \end{itemize}
\end{frame}

\end{document}