\documentclass[t,12pt,xcolor=dvipsnames]{beamer}
\usepackage{setspace}
\usepackage[canadian]{babel}
\setstretch{1.5}
\usepackage[utf8]{inputenc}
\usepackage{nicefrac}
\usepackage{biblatex}
\usetheme{Berlin}
\useoutertheme{infolines}
\useinnertheme{rounded}
\definecolor{darkergreen}{rgb}{0.12, 0.3, 0.17}
\definecolor{lapislazuli}{rgb}{0.15, 0.38, 0.61}
\definecolor{veige}{rgb}{0.12,,215}
\definecolor{black}{rgb}{0,0,0}
\definecolor{burlywood}{rgb}{0.87, 0.72, 0.53}
\setbeamercolor{background canvas}{bg=white}


\usecolortheme[named=black]{structure}
%\definecolor{cambridgeblue}{rgb}{0.64, 0.76, 0.68}
%\definecolor{lavenderblue}{rgb}{0.8, 0.8, 1.0}

\usepackage{amssymb,amsmath,tabu}
\usepackage{graphicx}
\usepackage{booktabs}
\usepackage{multirow}

\title[Applied \LaTeX]{Applied \LaTeX \ for Researchers}
\subtitle{Lecture 1: Introduction and Basics}
\author{Daniel Sánchez, M.A.}
\institute[LIDE]{Laboratorio de Investigación para el Desarrollo del Ecuador}
\date{Spring 2025}

\begin{document}
\begin{frame}
    \titlepage
\end{frame}

\begin{frame}{What is \LaTeX \ ? Why even care?}
    \begin{itemize}
    \item A typesetting system, widely used in academia.
    \item Allows for additional control over the structure and layout of documents other software does not easily provide.
    \item Free, open-source, and cross-platform.
    \item \textit{What you see is what you mean} (WYSIWYM) vs. \textit{What you see is what you get} (WYSIWYG).
    \item Allows for the creation and automation of complex, structured and consistent documents.
    \end{itemize}
\end{frame}

\begin{frame}{Outline}
    \tableofcontents
\end{frame}

\end{document}