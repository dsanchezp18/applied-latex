\documentclass[12pt]{article}

\title{\LaTeX \ para profesionales en datos \\[1em] \Large Temario}
\author{Daniel Sánchez}
\date{New Dimensions 2023 \\[1em] Laboratorio de Investigación para el Desarrollo del Ecuador}


\begin{document}

\maketitle

\subsection*{Introducción}

\begin{itemize}
    \item Introducción a \LaTeX, ¿qué es y para qué sirve?, ¿que veremos en la master class? 
    \item Breve historia
    \item Asuntos preliminares
    \item Instalación y ambientes de desarrollo
\end{itemize}

\subsection*{Creando documentos}

\begin{itemize}
    \item Estructura
    \item Comandos simples
    \item Paquetes 
    \item Clases de documentos
    \item Formateo básico y escritura
\end{itemize}

\subsection*{Modo matemático}

\begin{itemize}
    \item Modo matemático
    \item Símbolos básicos, letras griegas
    \item Ecuaciones
    \item Matrices
\end{itemize}

\subsection*{Tablas y figuras}

\begin{itemize}
    \item Tablas básicas
    \item Excel2LaTeX
    \item Inclusión de figuras
    \item Dibujar funciones
\end{itemize}

\subsection*{Integración con R}

\begin{itemize}
    \item *stargazer*, *kableExtra*, *modelsummary*
    \item Recap de RMarkdown y Quarto
    \item *sweave y knitr*
    \item *knitr* fuera de RStudio
\end{itemize}

\subsection*{Otras integraciones y misceláneos}

\begin{itemize}
    \item Python
    \item VS Code
    \item Git
    \item Word
    \item Software estadístico
    \item Beamer
    \item Mathpix y codecogs
    \item Bibliografía
\end{itemize}


\end{document}  