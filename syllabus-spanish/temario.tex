\documentclass[a4paper,12pt]{article}

\usepackage[utf8]{inputenc}
\usepackage[spanish]{babel}
\usepackage{hyperref}
\usepackage[margin=2cm]{geometry}
\usepackage[style=authoryear,backend=biber,citestyle=apa,maxnames=2]{biblatex}
\addbibresource{referencias_temario.bib}
\nocite{*}

\title{\textbf{\LaTeX \ Aplicado para Investigación Social Cuantitativa} \\[1em] \Large Descripción del Curso}
\author{Daniel Sánchez}
\date{Laboratorio de Investigación para el Desarrollo del Ecuador}

\begin{document}

\maketitle

\section{Resumen del curso}

Este curso introduce el uso del sistema tipográfico \LaTeX, enfócandose en su uso aplicado para la investigación en ciencias sociales. 

\section{Contenidos}

\subsection{Introducción}

\begin{itemize}
    \item Introducción a \LaTeX, ¿qué es y para qué sirve?, ¿que veremos en este curso? 
    \item Breve historia
    \item Asuntos preliminares
    \begin{itemize}
        \item Requisitos de hardware
        \item Instalación de distribución \TeX
        \item Ambientes de desarrollo
        \item Overleaf
        \item Identificación de teclas especiales de importancia
    \end{itemize}
   
\end{itemize}

\subsection{Creando documentos}

\begin{itemize}
    \item Estructura
    \item Comandos simples
    \item Paquetes 
    \item Clases de documentos
    \item Formateo básico del documento 
    \item Manejo de texto
    \begin{itemize}
        \item Alineación
        \item Listas 
        \item Títulos, portadas y resúmenes
        \item Índices
        \item Encabezados y notas al pie
    \end{itemize}
\end{itemize}

\subsection{Modo matemático}

\begin{itemize}
    \item Introducción a modo matemático
    \item Símbolos básicos, letras griegas
    \item Ecuaciones
    \item Matrices
\end{itemize}

\subsection{Tablas y figuras}

\begin{itemize}
    \item Tablas básicas
    \item Excel2LaTeX / Addins de Overleaf
    \item Inclusión de figuras y subfiguras
    \item Funciones matemáticas y gráficos con \textsf{tikz} y \textsf{pgfplots}
\end{itemize}

\subsection{Manejo de bibliografías y formateo APA}
\begin{itemize}
    \item Bib\TeX \ y Bib\LaTeX
    \item Integracion con Zotero
    \item Breve revisión de integraciones con Mendeley, Citavi y otros
    \item Formateo de citas y bibliografías
    \item Proyectos multiarchivo
    \item Formateo avanzado de documentos mediante \textit{clases}
\end{itemize}

\subsection{Integración con R}

\begin{itemize}
    \item Presentación de resultados de análisis de datos
        \begin{itemize}
            \item \textit{stargazer}
            \item \textit{kableExtra}
            \item \textit{modelsummary}
            \item \textit{gt}
        \end{itemize}
    \item R desde \LaTeX: \textit{knitr}
    \item \LaTeX \ con R: \textit{sweave} y \textit{knitr}, incluyendo bibliografías
\end{itemize}

\subsection{Otras integraciones y misceláneos}

\begin{itemize}
    \item Referenciación cruzada
    \item Control de versión de \LaTeX \ con Git
    \item Utilizando \LaTeX \ desde Word
    \item Integraciones con otro software de investigación
    \begin{itemize}
        \item Stata
        \item Python/Jupyter
        \item IBM SPSS
        \item MATLAB/Octave
        \item Mathematica
    \end{itemize}
    \item Presentaciones desde beamer
    \item Matemática más simple con mathpix y Codecogs
    \item VS Code como IDE de \LaTeX
    \item Consejos para escribir en \LaTeX \ y \textit{debugging} de errores
\end{itemize}

\printbibliography[title=Material de Referencia]


\end{document}  