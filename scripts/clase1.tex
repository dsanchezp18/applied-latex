% ----------------------------
% el preambulo comienza cuando surge documentclass
% documentclass "comienza" el preambulo

\documentclass[12pt, a4paper]{article} % el documento es un "article" != report
\usepackage{graphicx} % paquete para insertar imagenes
\usepackage{hyperref} % hipervinculos y secciones dinamicas. 
\usepackage[margin = 1in]{geometry} % modificar margenes
\usepackage[spanish]{babel} % soporte para espanol
\usepackage{amsmath} % para simbolos matematicos
\usepackage{lipsum} % texto de ejemplo
% ctrl + enter para correr mi codigo! "compilar"

% el preambulo termina cuando empieza el documento
% el ambiente document empieza con \begin{document}
% ----------------------------

% para borrar todas las sangrias:

\setlength{\parindent}{0cm} % poner el largo de la sangria de los parrafos a zero.
% no olvidar nunca unidades!!
% para borrar una sangria en especifico, \noindent.

% comandos de metadatos del documento:
\title{Clase 1: Basic \LaTeX}
\author{Daniel S \hspace{0.05cm} JC Munoz}
\date{\today}

\begin{document}

% generar el titulo:

\maketitle

\tableofcontents

\clearpage

\section{Resumen}
hola. 
\\ % doble backslash es un "linebreak", linea en blanco. 

El texto descriptivo, en este caso un retrato de una \textbf{persona}, provoca en el receptor una imagen tal que la \textit{realidad} descripta cobra forma, se materializa en su mente. En este caso el texto habla de un personaje real: Doña Uzeada de Ribera Maldonado de Bracamonte y Anaya. Como se trata de una descripcion literaria, la actitud del emisor es subjetiva, dado que pretende transmitir su propia vision personal al describir y la funcion del lenguaje es predominantemente poetica, ya que persigue una estetica en particular.

\large
esta letra es pequeña

este es un texto adicional

El texto descriptivo, en este caso un retrato de una \textbf{persona}, provoca en el receptor una imagen tal que la \textit{realidad} descripta cobra forma, se materializa en su mente. En este caso el texto habla de un personaje real: Doña Uzeada de Ribera Maldonado de Bracamonte y Anaya. Como se trata de una descripcion literaria, la actitud del emisor es subjetiva, dado que pretende transmitir su propia vision personal al describir y la funcion del lenguaje es predominantemente poetica, ya que persigue una estetica en particular.

\normalsize

tamano normal

\begin{tiny}
    este es un ambiente "tiny" en donde las letras son "tiny". 
\end{tiny}

salimos del ambiente y las letras son normal size. 

% textbf{} es para poner en negrillas
% text bold font.
% textit{} cursivas
% text italics
% underline{} es subrayar

el pib aumento en 10\%

esto es un backslash: \textbackslash

\section{Introduction}

Esta es mi seccion de introduccion. 

\subsection{Subseccion 1}

\lipsum[5] % texto de ejemplo, el corchete denota numero de parrafos.

\section{Listas}

Dos tipos de listas: 

% enumeracion:
\begin{enumerate}
    \item Enumeracion
    \begin{enumerate}
        \item numero 1
        \item numero 2
    \end{enumerate}
    \item Items
\end{enumerate}

% itemizacion:

\begin{itemize}
    \item item 1
        \begin{itemize}
            \item otro subitem
            \item subitem
        \end{itemize}
    \item item 2
    \begin{enumerate}
        \item numerado
    \end{enumerate}
\end{itemize}

\section*{Seccion no numerada}

\end{document}
% ----------------------------
% terminar documento con \end{document}

