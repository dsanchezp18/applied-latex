% This is the "preamble" of the TeX file. 
\documentclass[12pt]{article} % document class gives base settings to the document

% functions take in arguments. 
% "mandatory" arguments
% are entered with curly brackets. 
% optional arguments
% square brackets, before the curly brackets. 

\usepackage{graphicx} % Required for inserting images
\usepackage{lipsum} % package for dummy text
\usepackage{setspace} % changes spacing for lines
    \doublespacing % set double spacing for the complete document
\usepackage{hyperref} % for hyperlinks and section links
\usepackage{amsmath} % for math stuff
\usepackage{babel}
\usepackage{booktabs} % for table lines

\title{Class} % these are metadata functions, they do not create the title. 
\author{Daniel Sánchez}
\date{March 2025}


\begin{document} % the preamble ends after the begin document function

\maketitle % make the title command

\tableofcontents

Hello, this is some text.

I can write bold as \textbf{bold}. 

\textit{italics}

\underline{underline}.

This is a list. 

\begin{itemize}
    \item  This is the first item in my list.
    \item This is the second item in my list.
\end{itemize}

\begin{enumerate}
    \item First item
    \begin{enumerate}
        \item a nested item 
        \item another nested item
    \end{enumerate}
    \item some other item
    \item item
    \item item 
\end{enumerate}

\noindent \lipsum[1]

\vspace{0.5cm}

\lipsum[2]

% never forget to use the percentage function if you actually want to write a percentage in-text. 

GDP increased by 10\% and inflation rose by 5\%.

\textbackslash

\section{Introduction}

\lipsum 

\subsection{This is a subsection}

\lipsum 

\subsubsection{This subsubsection}

\lipsum 

\section{Another section}

\lipsum[2]

\section*{Section that is not numbered}

\section{Math Mode}

Enter math mode with the dollar sign:

$x + 1$

x + 1

\textit{x+1}

$x^2$

$x_2$

$\frac{2}{3}$
Rules of math mode:

\begin{itemize}
    \item Powers are hats.
    \item Subscripts are underscores
    \item Fractions are dfrac
\end{itemize}

This is a line. It has math in it. $x + 1 = 2$

This is another line with display math $$x + 1 = 2$$

The equation environment:

\begin{equation}
    x^2 + x^3 + \frac{3}{5}
    \label{eqn:equation}
\end{equation}

Equation \ref{eqn:equation} is the first equation in my paper. 

This is a sentence with a fraction $\frac{2}{3}$. 

This is another sentence with a fraction $\dfrac{2}{3}$. 

Greek letters come with the ams math package. 

\begin{equation*}
     y = \beta_0 + \beta_1 x + \varepsilon
\end{equation*}

To create a summation with limits, use the underscore and exponent:

$$ \sum_{i=1}^n x_i $$

$$\int_{x}^{y} x \ dx$$

\begin{equation}
    \text{earnings} = \beta_0 + \beta_1 \text{school} + \epsilon 
\end{equation}

$$log(y)$$

$$\log_5(y)$$

$$\ln x$$

$$ \max {y} $$

$$ \min{y} $$

\begin{align*}
&x + 6 = 10  && &x - 7 \qquad  = 9\\ 
&y + 1 + x= 20  && &y + 8  \quad = 9\\
&z - 7  = 20 && &z + 6  = 0
\end{align*}


\begin{table}[h]
    \centering
    \begin{tabular}{cccr}
    \toprule
        A & b & c & D \\
    \midrule
        1 & 2 & 3 & 6 \\
        4 & 5 & 6 & 7 \\
    \bottomrule
    \end{tabular}
    \caption{Caption}
    \label{tab:my_label}
\end{table}


\begin{table}[]
\begin{tabular}{|l|c|}
\hline
                                                               & (1)                   \\ \hline
VARIABLES                                                      & turnout\_total        \\ \hline
                                                               &                       \\ \hline
treatment                                                      & 8.301**               \\ \hline
                                                               & (3.487)               \\ \hline
registered\_total                                              & 0.0582***             \\ \hline
                                                               & (0.00382)             \\ \hline
Constant                                                       & 337.2***              \\ \hline
                                                               & (9.714)               \\ \hline
                                                               &                       \\ \hline
Observations                                                   & 6,950                 \\ \hline
R-squared                                                      & 0.111                 \\ \hline
Standard   errors in parentheses                               & \multicolumn{1}{l|}{} \\ \hline
***   p\textless{}0.01, ** p\textless{}0.05, * p\textless{}0.1 & \multicolumn{1}{l|}{} \\ \hline
\end{tabular}
\end{table}

\end{document}
